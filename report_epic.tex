% Options for packages loaded elsewhere
\PassOptionsToPackage{unicode}{hyperref}
\PassOptionsToPackage{hyphens}{url}
%
\documentclass[
]{article}
\title{Epic data Results}
\author{C Ryan King, MD PhD}
\date{June 20, 2022}

\usepackage{amsmath,amssymb}
\usepackage{lmodern}
\usepackage{iftex}
\ifPDFTeX
  \usepackage[T1]{fontenc}
  \usepackage[utf8]{inputenc}
  \usepackage{textcomp} % provide euro and other symbols
\else % if luatex or xetex
  \usepackage{unicode-math}
  \defaultfontfeatures{Scale=MatchLowercase}
  \defaultfontfeatures[\rmfamily]{Ligatures=TeX,Scale=1}
\fi
% Use upquote if available, for straight quotes in verbatim environments
\IfFileExists{upquote.sty}{\usepackage{upquote}}{}
\IfFileExists{microtype.sty}{% use microtype if available
  \usepackage[]{microtype}
  \UseMicrotypeSet[protrusion]{basicmath} % disable protrusion for tt fonts
}{}
\makeatletter
\@ifundefined{KOMAClassName}{% if non-KOMA class
  \IfFileExists{parskip.sty}{%
    \usepackage{parskip}
  }{% else
    \setlength{\parindent}{0pt}
    \setlength{\parskip}{6pt plus 2pt minus 1pt}}
}{% if KOMA class
  \KOMAoptions{parskip=half}}
\makeatother
\usepackage{xcolor}
\IfFileExists{xurl.sty}{\usepackage{xurl}}{} % add URL line breaks if available
\IfFileExists{bookmark.sty}{\usepackage{bookmark}}{\usepackage{hyperref}}
\hypersetup{
  pdftitle={Epic data Results},
  pdfauthor={C Ryan King, MD PhD},
  hidelinks,
  pdfcreator={LaTeX via pandoc}}
\urlstyle{same} % disable monospaced font for URLs
\usepackage[margin=1in]{geometry}
\usepackage{longtable,booktabs,array}
\usepackage{calc} % for calculating minipage widths
% Correct order of tables after \paragraph or \subparagraph
\usepackage{etoolbox}
\makeatletter
\patchcmd\longtable{\par}{\if@noskipsec\mbox{}\fi\par}{}{}
\makeatother
% Allow footnotes in longtable head/foot
\IfFileExists{footnotehyper.sty}{\usepackage{footnotehyper}}{\usepackage{footnote}}
\makesavenoteenv{longtable}
\usepackage{graphicx}
\makeatletter
\def\maxwidth{\ifdim\Gin@nat@width>\linewidth\linewidth\else\Gin@nat@width\fi}
\def\maxheight{\ifdim\Gin@nat@height>\textheight\textheight\else\Gin@nat@height\fi}
\makeatother
% Scale images if necessary, so that they will not overflow the page
% margins by default, and it is still possible to overwrite the defaults
% using explicit options in \includegraphics[width, height, ...]{}
\setkeys{Gin}{width=\maxwidth,height=\maxheight,keepaspectratio}
% Set default figure placement to htbp
\makeatletter
\def\fps@figure{htbp}
\makeatother
\setlength{\emergencystretch}{3em} % prevent overfull lines
\providecommand{\tightlist}{%
  \setlength{\itemsep}{0pt}\setlength{\parskip}{0pt}}
\setcounter{secnumdepth}{5}
\usepackage{colortbl}
\usepackage{amsmath}
\usepackage{colortbl}
\ifLuaTeX
  \usepackage{selnolig}  % disable illegal ligatures
\fi

\begin{document}
\maketitle

\hypertarget{methods}{%
\subsection{Methods}\label{methods}}

Data from 2018-08-31 to 2020-12-30 were queried from the Epic electronic health record.
Patients were matched to hospital admission, discharge, billing and mortality data.
Records were filtered to include those with preoperative clinic visits with AD8 or SBT measured, surgery within 90 days of that clinic visit, and matching billing and discharge records.
A subset of surgical procedures were included into the analytical dataset, procedure codes for which are in the appendix. This restriction was intended to reduce confounding due to surgical complexity.
That is, we suspect that patients with impaired cognition may be selectively referred to medical vs surgical management of their conditions, and that patients with impaired cogition will be referred for surgeries with less physiologic stress.
These procedures were felt to be ones with little variation in overall surgical complexity which would contribute to this selection bias.

Because our principal outcomes (readmission, discharge to home) and inclusion critereon (procedure codes) are generated at the hospitalization level, we treat that as the unit of analysis.
Readmission in 30 days and length of stay was only considered for patients with discharge to home.

Association between preoperative cognitive impairment (defined as a Short Blessed Test \textgreater= 2 or a AD8 \textgreater= 5) was assesed using logistic regression models adjusting for surgical procedure type, sex, age using a 5 degree of freedom cublic spline, indicator variables for history of diabetes, chronic kidney disease, chronic obstructive pulmonary disease, CVA(TIA), cancer status, and congestive heart failure.\\
Each hospitalization can have procedures qualifying in multiple categories; we created indicator variables for each procedure category and assumed additivity for hospitalizations with more than one type of procedure.
Logisitic regression models with the same adjustment startegy were considered for readmission and near-term death.
A log-link generalized linear model (quasipoisson) with the same set of adjusting variables was used for length of stay.

Heterogeneity of association between abnormal cognition and outcomes was assessed using an expanded model with an interaction term and a score-test.
To compare the predictive value of AD8 or SBT, we used two approaches.
First, we used a k-fold cross-validation method (K=100) fitting logistic regression models predicting each outcome using AD8 or SBT, then assessing the accuracy by area under the receiver operating characteristic curve in the hold-out sample. Each 100 pairs of accuracy metrics are then compared by a paired t-test.
Second, we used the Vuong test for non-nested models \url{https://doi.org/10.2307/1912557} implemented by the ``nonnest2'' package version 0.5-5.

All analysis was conducted using R 4.1.2. A repository containing the analysis code is available at \url{https://github.com/cryanking/cognition_discharge}

\hypertarget{results}{%
\subsection{Results}\label{results}}

The filtered dataset included 8315 distinct patients with 8950 hospitalizations.
Characteristics of the cohort and overall outcome rates are given in Table \ref{tab:desc}.
AD8 was abnormal (\textgreater=2) in 3.7 percent of included patients; SBT was abnormal (\textgreater=5) in 17.8 percent of included patients.

\begin{longtable}[t]{l|l|l|l}
\caption{\label{tab:tableone}\label{tab:desc}Descriptive statistics stratified by Cognition status. P-values for quantitative variables by Mann-Whitney U, factor variables by Fisher's exact test. CAD = Coronary artery disease, CHF = Congestive heart failure,  Diabetes = Diabetes mellitus,  COPD = Chronic obstructive pulmonary diseae, CKD = chronic kidney disease, CVA\(TIA) = history of stroke or transient ischemic attack,  dc\_home = discharge to home, readmit  = readmission within 30 days, los = postoperative hospital length of stay computed from first qualifying procedure, death = death in hospistal or within 30 days of surgery}\\
\hline
  & normal & impaired cognition & p\\
\hline
\endfirsthead
\caption[]{\label{tab:tableone}Descriptive statistics stratified by Cognition status. P-values for quantitative variables by Mann-Whitney U, factor variables by Fisher's exact test. CAD = Coronary artery disease, CHF = Congestive heart failure,  Diabetes = Diabetes mellitus,  COPD = Chronic obstructive pulmonary diseae, CKD = chronic kidney disease, CVA\(TIA) = history of stroke or transient ischemic attack,  dc\_home = discharge to home, readmit  = readmission within 30 days, los = postoperative hospital length of stay computed from first qualifying procedure, death = death in hospistal or within 30 days of surgery \textit{(continued)}}\\
\hline
  & normal & impaired cognition & p\\
\hline
\endhead
\cellcolor{gray!6}{\cellcolor{gray!6}{n}} & \cellcolor{gray!6}{\cellcolor{gray!6}{7128}} & \cellcolor{gray!6}{\cellcolor{gray!6}{1822}} & \cellcolor{gray!6}{\cellcolor{gray!6}{}}\\
\hline
Sex = 2 (\%) & 3530 (49.5) & 993 (54.5) & <0.001\\
\hline
\cellcolor{gray!6}{\cellcolor{gray!6}{RACE (\%)}} & \cellcolor{gray!6}{\cellcolor{gray!6}{}} & \cellcolor{gray!6}{\cellcolor{gray!6}{}} & \cellcolor{gray!6}{\cellcolor{gray!6}{NA}}\\
\hline
\hspace{1em}Other & 48 ( 0.7) & 17 ( 0.9) & \\
\hline
\hspace{1em}\cellcolor{gray!6}{\cellcolor{gray!6}{White}} & \cellcolor{gray!6}{\cellcolor{gray!6}{6392 (89.7)}} & \cellcolor{gray!6}{\cellcolor{gray!6}{1436 (78.8)}} & \cellcolor{gray!6}{\cellcolor{gray!6}{}}\\
\hline
\hspace{1em}Black & 637 ( 8.9) & 357 (19.6) & \\
\hline
\hspace{1em}\cellcolor{gray!6}{\cellcolor{gray!6}{Asian}} & \cellcolor{gray!6}{\cellcolor{gray!6}{45 ( 0.6)}} & \cellcolor{gray!6}{\cellcolor{gray!6}{11 ( 0.6)}} & \cellcolor{gray!6}{\cellcolor{gray!6}{}}\\
\hline
\hspace{1em}other\_pacific\_islands & 6 ( 0.1) & 1 ( 0.1) & \\
\hline
\cellcolor{gray!6}{\cellcolor{gray!6}{CAD (\%)}} & \cellcolor{gray!6}{\cellcolor{gray!6}{1856 (26.0)}} & \cellcolor{gray!6}{\cellcolor{gray!6}{639 (35.1)}} & \cellcolor{gray!6}{\cellcolor{gray!6}{<0.001}}\\
\hline
CHF (\%) & 830 (11.6) & 379 (20.8) & <0.001\\
\hline
\cellcolor{gray!6}{\cellcolor{gray!6}{Diabetes (\%)}} & \cellcolor{gray!6}{\cellcolor{gray!6}{2115 (29.7)}} & \cellcolor{gray!6}{\cellcolor{gray!6}{684 (37.5)}} & \cellcolor{gray!6}{\cellcolor{gray!6}{<0.001}}\\
\hline
COPD (\%) & 1189 (16.7) & 423 (23.2) & <0.001\\
\hline
\cellcolor{gray!6}{\cellcolor{gray!6}{CKD (\%)}} & \cellcolor{gray!6}{\cellcolor{gray!6}{1646 (23.1)}} & \cellcolor{gray!6}{\cellcolor{gray!6}{603 (33.1)}} & \cellcolor{gray!6}{\cellcolor{gray!6}{<0.001}}\\
\hline
CVA(TIA) (\%) & 447 ( 6.3) & 276 (15.1) & <0.001\\
\hline
\cellcolor{gray!6}{\cellcolor{gray!6}{cancerStatus (\%)}} & \cellcolor{gray!6}{\cellcolor{gray!6}{}} & \cellcolor{gray!6}{\cellcolor{gray!6}{}} & \cellcolor{gray!6}{\cellcolor{gray!6}{NA}}\\
\hline
\hspace{1em}Skin Cancer & 4646 (65.2) & 1224 (67.2) & \\
\hline
\hspace{1em}\cellcolor{gray!6}{\cellcolor{gray!6}{in remission/radiation/chemo}} & \cellcolor{gray!6}{\cellcolor{gray!6}{901 (12.6)}} & \cellcolor{gray!6}{\cellcolor{gray!6}{193 (10.6)}} & \cellcolor{gray!6}{\cellcolor{gray!6}{}}\\
\hline
\hspace{1em}Current Cancer & 1388 (19.5) & 339 (18.6) & \\
\hline
\hspace{1em}\cellcolor{gray!6}{\cellcolor{gray!6}{Metastatic Cancer}} & \cellcolor{gray!6}{\cellcolor{gray!6}{193 ( 2.7)}} & \cellcolor{gray!6}{\cellcolor{gray!6}{66 ( 3.6)}} & \cellcolor{gray!6}{\cellcolor{gray!6}{}}\\
\hline
age (median [IQR]) & 72 [68, 76] & 74 [69, 79] & <0.001\\
\hline
\cellcolor{gray!6}{\cellcolor{gray!6}{dc\_home (\%)}} & \cellcolor{gray!6}{\cellcolor{gray!6}{1139 (16.0)}} & \cellcolor{gray!6}{\cellcolor{gray!6}{662 (36.3)}} & \cellcolor{gray!6}{\cellcolor{gray!6}{<0.001}}\\
\hline
readmit (\%) & 318 ( 4.5) & 90 ( 4.9) & 0.379\\
\hline
\cellcolor{gray!6}{\cellcolor{gray!6}{death (\%)}} & \cellcolor{gray!6}{\cellcolor{gray!6}{164 ( 2.3)}} & \cellcolor{gray!6}{\cellcolor{gray!6}{109 ( 6.0)}} & \cellcolor{gray!6}{\cellcolor{gray!6}{<0.001}}\\
\hline
ICU (\%) & 1228 (17.2) & 426 (23.4) & <0.001\\
\hline
\cellcolor{gray!6}{\cellcolor{gray!6}{los (median [IQR])}} & \cellcolor{gray!6}{\cellcolor{gray!6}{4 [2, 9]}} & \cellcolor{gray!6}{\cellcolor{gray!6}{5 [2, 11]}} & \cellcolor{gray!6}{\cellcolor{gray!6}{<0.001}}\\
\hline
\end{longtable}

The frequency of each type of surgery is displayed in \ref{tab:descType}. The frequencies do not add up to the number of hospitalizations because of multiple procedures per hospitalization.

\begin{table}

\caption{\label{tab:tabletwo}\label{tab:descType}Number of Procedures by Type}
\centering
\begin{tabular}[t]{l|r}
\hline
Surgery Type & N\\
\hline
AV fistula & 7\\
\hline
VATS & 49\\
\hline
lap hiatal hernia & 56\\
\hline
cholecystectomy & 81\\
\hline
cystectomy & 92\\
\hline
nephrectomy & 156\\
\hline
gastric & 168\\
\hline
pancreatic & 174\\
\hline
prostatectomy & 294\\
\hline
lumbar fusion & 406\\
\hline
total hip & 417\\
\hline
total knee & 480\\
\hline
hysterectomy & 491\\
\hline
total shoulder & 491\\
\hline
intestinal & 662\\
\hline
\end{tabular}
\end{table}

We found that impaired cognition was significantly associated with discharge to home (odds ratio 2.48, 95\% CI 2.19 to 2.81, p = \textless0.001).
For readmission in 30 days, impaired cognition was not a significant predictor (odds ratio 0.90, 95\% CI 0.65 to 1.25, p = 0.168).
Similarly for near-term death, impaired cognition was a significant predictor (odds ratio 1.93, 95\% CI 1.48 to 2.51, p = \textless0.001).
Similarly, impaired cognition was not a significant predictor of length of stay (duration ratio 1.04, 95\% CI 0.91 to 1.06, p = 0.611).

We found modest evidence of heterogeneity by type of surgery, although power for this question is limited by small sample sizes in some surgery types. A score test comparing models with and without interaction terms yielded a p-value of 0.370
A forest plot is presented in Figure \ref{fig:fig1}.

\begin{figure}
\centering
\includegraphics{/research/forest_home_epic.png}
\caption{\label{fig:figone}\label{fig:fig1}Odd-ratio for discharge to home with impaired cogition by surgery type. Dot = point estimate, lines 95 \% confidence intervals}
\end{figure}

\end{document}
